\typeappendix{true}
\chaptersubtitle{Brayan Riveros}
\chapter{Estándares}\label{cap:estandares}
Antes de mencionar los estándares del código que deben usarse con esta plantilla, tenga en cuenta que no son obligatorios, aun así veamos que es la notación Camel Case\pap 
Camel case es una notación que nos permite nombrar las variables, funciones o clases en un lenguaje de programación, esta a su vez se divide en dos casos UpperCamelCase y lowerCamelCase.\pap
\textbf{UpperCamelCase}.
La escritura utilizando UpperCamelCase es iniciar en mayúscula cada palabra que compone el nombre de la variable, sin olvidar que esta no puede tener espacios ni caracteres especiales.\pap 
\textbf{lowerCamelCase}.
La escritura utilizando lowerCamelCase es iniciar en minúscula cada palabra que compone el nombre de la variable, sin olvidar que esta no puede tener espacios ni caracteres especiales.
\section{Comandos}
Recuerde que la estructura de un comando es:
\begin{center}
	\begin{tcblisting}{boxlatex}
		\newcommand{\nombreDeComando}[<arguments number>]{<function>}
	\end{tcblisting}
\end{center}
Para nombrar correctamente los comandos se usará la notación lowerCamelCase
\subsubsection{Referencias}
Todas las referencias de la mayor parte de los entornos tienen un prefijo, esto conlleva a una buena organización y previene el conflicto de referencias iguales para entornos diferentes.
\begin{freebox}[Convención]
	Algunas referencias no dependen da la plantilla como lo es el caso de los capítulos y ecuaciones, por ello se recomienda seguir el siguiente  estándar: 
	\begin{itemize}
		\item Capítulo. \verb|\label{chapter:<nombre>}|.
		\item Sección. \verb|\label{sec:<nombre>}|
		\item Ecuación. \verb|\label{eqn:<nombre>}|.
		\item Item. \verb|\label{itm:<nombre>}|.
		\item Imagen. \verb|\label{fig:<nombre>}|.
		\item Bibliografía. \verb|\label{bib:<nombre>}| o \verb|\label{book:<nombre>}|.
		\item Tabla. \verb|\label{tab:<nombre>}|.
	\end{itemize}
	\begin{freebox}[Nota]
		Para el \verb|<nombre>| utilizaremos la estandarización lowerCamelCase. 
	\end{freebox}
\end{freebox}
\subsubsection{Estándar de nombres de archivos}
Para los nombres de todos los archivos del autor (incluyendo imágenes y código \LaTeX) no se utilizará espacios ni caracteres especiales, entonces seguiremos el estándar lowerCamelCase, por ejemplo 
\begin{center}
	\textbf{algebraLineal.tex} o \textbf{cuboRubik.png.}
\end{center}
\begin{freebox}[Nota]
	Los nombres de los archivos utiliza el idioma en el que esta redactado el documento.
\end{freebox}
\newpage
\section{Sangrado (Indentación)}
El contenido de los entornos debe iniciar con una tabulación, así como se muestra en el siguiente código:
\begin{tcblisting}{boxlatex}
	\begin{<entorno>}
		Contenido del entorno.
	\end{<entorno>}
\end{tcblisting}
De forma similar en las ecuaciones se realizan las respectivas tabulaciones en el símbolo \& tal como se observa en el siguiente código: 
\begin{tcblisting}{boxlatex}	
	\begin{align*}
		(a+b)c&=(a+b)\cdot c\\
			    &=ac+bc.
	\end{align*}
\end{tcblisting}