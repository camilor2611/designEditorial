{
	\justifying
	\chapterauthor{Brayan Riveros}%Puede ser nombre simplificado
	\chapterimage{jpg/robertOppenheimer.jpg}{3cm}
	\chapter{Guía de usuario}\label{cap:guiaDeUsuario}
	\epigraph{Now, I am become Death, the destroyer of worlds.}{--- \textup{Robert Oppenheimer}}
	\section{Estructura del autor}
		\lettrine[lines=2, nindent=0em]{U}{na} vez realizada la instalación de la plantilla 
		y siguiendo con nuestro ejemplo de la creación del artículo \comillas{ecuación de difusión} cuya carpeta es \textbf{ecuacionDeDifusion} se tiene las siguientes lineas en el archivo \textbf{main.tex}
		\pap
		\begin{tcblisting}{boxlatex}
			\def\ecuacionDeDifusion{articles/ecuacionDeDifusion}
			\def\ecuacionDeDifusionImages{\ecuacionDeDifusion/images}
		\end{tcblisting}
	  	Estos comandos nos sirven para la ruta del artículo y la ruta de las imágenes que usará el autor respectivamente. También consideramos la estructura básica con los siguientes archivos del autor.
		\begin{figure}[ht]
			\dirtree{%
				.1 {\color{colorPer1}\faFolder} CopiaDeTrabajo.
				.2 {\color{colorPer1}\faFolder} articles.
				.3 {\color{colorPer1}\faFolder} ecuacionDeDifusion.
				.4 {\color{colorPer1}\faFolder} images.
				.5 {\color{colorPer1}\faFolder} png.
				.6 {\color{colorPer1}\faImage} robertOppenheimer.png.
				.4 {\color{colorPer1}\faFolder} src.
				.4 {\color{colorPer1}\faFileCodeO} article.tex.
				.4 {\color{colorPer1}\faFileCodeO} load.sty.
				.2 {\color{colorPer1}\faFolder} DesignEditorial10.
				.2 {\color{colorPer1}\faFolder} images.
				.2 {\color{colorPer1}\faFolder} references.
				.3 {\color{colorPer1}\faFileCodeO} ecuacionDeDifusion.bib.
				.2 {\color{colorPer1}\faFolder} settings.
				.2 {\color{colorPer1}\faFolder} src.
				.2 {\color{colorPer1}\faFileCodeO} main.tex.
			}	
			\caption{Estructura de carpetas del autor}
			\label{fig:EstructuraDeAutor}
		\end{figure}
	\section{Artículo personalizado}
	Antes de iniciar la redacción de cualquier artículo entendamos la estructura básica del mismo, el archivo \textbf{article.tex} debe ser similar a
	\begin{tcblisting}{boxlatex}
	{\justifying
		\chapterauthor{<Autor>}
		\chapter{<nombre del articulo>}\label{art:<nombre upperCamelCase>}
	}\cleanalldata
	\end{tcblisting}
	Los siguientes datos son los que se mostrarán en el encabezado del artículo
	\begin{center}
		<autor>  = Nombre del autor, como guste el autor.\\  
		<nombre del articulo> = Es el nombre del artículo completo.\\
	\end{center}
	El comando \verb|\label| es la referencia del artículo y debe ser escrito en lowerCamelCase (ver apartado \ref{cap:estandares} estándares).\pap
	Así que siguiendo con nuestro ejemplo tendríamos algo de esta forma
	\begin{tcblisting}{boxlatex}
		{\justifying
			\chapterauthor{Brayan Riveros}
			\chapter{Ecuación de difusión}\label{art:ecuacionDeDifusion}
		}\cleanalldata
	\end{tcblisting}
%	\putbib % muestra la bibliografia
}
