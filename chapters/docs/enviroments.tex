{
	\justifying
	\chapter{Entornos}\label{cap:guiaDeUsuario}
	Los siguientes entornos tienen dos argumentos obligatorios, si no se desea utilizar los argumentos se deben dejar en blanco, esto es \{\}\{\}. El primer argumento es el nombre del entorno y el segundo es la llave (key) de la referencia, vemos los entornos más usuales. 
	\begin{tcblisting}{boxlatex}
		\begin{theorem}{Lema de Zorn}{lemaZorn}
			Contenido lema
		\end{theorem}
		Para referenciar el lema se utiliza \ref{lem:lemaZorn}
	\end{tcblisting}
	Ejemplo
	\begin{theorem}{Teorema fundamental del Cálculo}{TFC}
		Contenido
	\end{theorem}
	Según teorema	\ref{th:TFC} obtenemos ...
	
	\begin{tcblisting}{boxlatex}
		\begin{lemma}{Lema de Zorn}{lemaZorn}
			Contenido lema
		\end{lemma}
		Para referenciar el lema se utiliza \ref{lem:lemaZorn}
	\end{tcblisting}
	Ejemplo
	\begin{lemma}{Lema de Zorn}{lemaZorn}
		Contenido lema
	\end{lemma}
	Según el lema \ref{lem:lemaZorn} obtenemos ...
	\begin{proof}
		Sea ...
	\end{proof}
	
	\begin{example}{}{}
		\lipsum[1]
	\end{example}

	\begin{solution}
		\lipsum[1]
	\end{solution}
}
